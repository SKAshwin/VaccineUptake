\begin{minipage}{6.25in}
	\centering
	\def\sym#1{\ifmmode^{#1}\else\(^{#1}\)\fi}
	\begin{tabular*}{\textwidth}{@{\extracolsep{\fill}} l*{4}{c}}
	\hline\hline
	            &\multicolumn{1}{c}{(1)}&\multicolumn{1}{c}{(2)}&\multicolumn{1}{c}{(3)}&\multicolumn{1}{c}{(4)}\\
	            &\multicolumn{1}{c}{Two Doses}&\multicolumn{1}{c}{One Dose}&\multicolumn{1}{c}{$\Delta \textrm{Two Doses}$}&\multicolumn{1}{c}{$\Delta \textrm{One Dose}$}\\
	\hline
	treated     &      0.0315         &       1.696         &     0.00712         &      0.0332         \\
	            &     (0.116)         &     (1.814)         &    (0.0239)         &    (0.0187)         \\
	\hline
	\(N\)       &        1178         &        1178         &        1178         &        1065         \\
	\hline\hline
	\end{tabular*}
	\caption*{\footnotesize{Notes: Standard errors in parentheses, clustered by state and county-pair. Column 1 shows the estimated treatment effect on the percentage of people who received two doses of the vaccine, within seven days of the start of the treatment. Column 2 shows the estimated treatment effect on the percentage of people who received at least one dose of the vaccine. Columns 3 and 4 measure the effect of the treatment on the daily change in the percentage of people who received both doses and one dose of the vaccine respectively. County fixed effects and pair-time fixed effects are not reported, for brevity. It is worth noting that, as would be expected, the effect of the lottery on the uptake of first doses is substantially higher the effect on the uptake of second doses, given that only the former enters you into the lottery and we consider only the first 10 days after the lottery began, before those who just got their first doses for the lottery would get a second dose. However, the effects are still not statistically different from 0.\\
		\sym{*} \(p<0.05\), \sym{**} \(p<0.01\), \sym{***} \(p<0.001\)}}
\end{minipage}

