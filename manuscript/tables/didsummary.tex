\begin{minipage}{6.25in}
	\centering
	\def\sym#1{\ifmmode^{#1}\else\(^{#1}\)\fi}
	\def\arraystretch{1.1}
	\begin{tabular}{l*{4}{cc}}
	\hline\hline
	            &\multicolumn{1}{c}{(1)}&\multicolumn{1}{c}{(2)}&\multicolumn{1}{c}{(3)}&\multicolumn{1}{c}{(4)}\\
	            &     Treated&   Untreated&Pair Differences&Nationwide SD\\
	\hline
	repvotes&       72.87&       80.46&        8.65&       16.03\\
	black       &        0.82&        0.72&        0.67&       14.79\\
	fullcollege &       24.75&       24.51&        7.12&        9.54\\
	casespc&        0.04&        0.06&        0.03&        0.02\\
	whiteevangelical&       31.79&       32.67&        6.65&       12.64\\
	catholic    &       13.79&       15.89&        4.45&       10.03\\
	poverty     &       12.81&       11.74&        3.13&        5.78\\
	medinc&    68923.79&    69379.73&    11087.17&    16720.16\\
	pop60to79   &       21.72&       21.87&        3.61&        4.58\\
	above80     &        5.37&        5.98&        1.75&        1.50\\
	\hline
	\(N\)       &          14&          18&          62&        3075\\
	\hline\hline
	\end{tabular}
	\caption*{\footnotesize{Columns 1 and 2 report the average value of the predictive covariates in the treated and untreated groups, corresponding to the Colorado border counties and their neighbors respectively (see Figure \ref{fig:bordercounties}). Column 3 contains the average absolute difference in covariates between two counties along the Colorado border which are connected. Column 4 is the standard deviation of the covariates across the entire nationwide dataset. Covariates are selected as those with the largest standardized effect sizes in Table \ref{table:crosssection}.}}
\end{minipage}
