\documentclass[12pt]{article}
\title{Part IIA Paper 3 Project}
\author{
	Anonymous
}

\newcommand\wordcount{
	\immediate\write18{texcount -sum -1 \jobname.tex > count.txt} 
	\input{count.txt}
}

\usepackage[a4paper, total={6.25in, 10in}]{geometry} % to adjust the margins etc
\usepackage{titlesec} % for the title formatting
\usepackage{setspace} % to set line spacing
\usepackage[parfill]{parskip} % remove the paragraph indentation
\usepackage{caption} % used to allow smaller "notes" under graphs/tables besides the main caption
\usepackage{booktabs} % used for the lines under the headers in the tables
\usepackage{multirow} % to allow multiple row option in table formatting
\usepackage{array} % for struts in formatting table
\usepackage{fancyhdr} % for moving page number to the bottom right corner
\usepackage{amsmath} % for multi line equations aligning
\usepackage{bm} % for bolded math symbols (for vectors)
\usepackage{graphicx}

\titleformat*{\section}{\centering \LARGE}
\pagenumbering{arabic} % add page numbers
% next few lines are to get the page numbers to the bottom right
\pagestyle{fancy}
\fancyhf{}
\renewcommand{\headrulewidth}{0pt}
\fancyfoot[R]{\thepage}


\begin{document}
	\maketitle
	\begin{spacing}{1.5} % set 1.5 line spacing
		Q1. Vaccination take-up rates: Inter-State differences in the USA in 2021.
		
		At the end of 2021 the share of the population fully vaccinated against Covid-19 differed widely across US States.
		
		a) Identify and evaluate the main economic and social factors giving rise to this outcome.
		
		b) On the basis of your answer to a), suggest how high vaccination rates might be achieved across the entire United States.
		
		\wordcount words
		\section{Introduction}
		Hello world hi hi hello d
		
		By the end of 2021, only 2\% of people wanted a vaccine and had not yet got it, from survey
		
		\section{Data and Methods}
		\subsection{Data}
		Regressions were generally done on the county level. Data on vaccination rates across time by county (and by state) were all sourced from the CDC (CITE). A dataset of time-invariant characteristics of each county was also compiled from many sources for 3075 counties in the United States; remaining counties were excluded due to some or all information not being available for them. This includes all US territories (such as Puerto Rico), which do not have presidential voting records, and all counties in Alaska and Hawaii. Utah's counties were also excluded due to spotty COVID-19 case data. Outside of these states, the following counties also had missing records: Bedford and Clifton Forge City, Dukes, Nantucket, Barnstable, and Ogala Lakota County, and Yellowstone National Park (which has no population).
		
		2020 election results by county were obtained from the MIT Election Data and Science Lab's Election Returns Dataverse (CITE). Religious breakdowns by county were obtained from the Public Religion Research Institute (PRRI) 2020 census of American Religion (CITE). Education attainment by county, averaged over 2015-2019, was obtained from the US Census Bureau ('the Census'), via the 2015-2019 American Community Survey (CITE). Poverty rates by county in 2019 were obtained from the Census' Small Area Income and Poverty Estimates, which also contained the 2013 Rural-Urban Continuum Code assigned to each county, which classifies counties based on their level of urbanization and proximity to metropolitan areas (CITE explanation). Racial breakdowns by county were obtained from the 2020 Decennial Redestricting Dataset, maintained by the Census (CITE). Age breakdowns by county were obtained from the County Characteristics Resident Population Estimates maintained by the Census (CITE). Median family incomes and costs of living for a family of four were obtained from the Economic Policy Institute's (EPI) Family Budget Calculatior (CITE).  COVID-19 Cases recorded in each county, as of 1 December 2020, were obtained from the CSSE COVID-19 Data Repository maintained by John Hopkins (CITE). Finally, data mapping each county to every county it borders was obtained (the 'county adjacency dataset') from a dataset maintained by the National Bureau of Economic Research (CITE), originating from a dataset maintained by the Census. 
		
		All datasets tagged each county with their Federal Information Processing Standard Publication (FIPS) codes, which was used to merge them together.
		% commentary on different years for data
		\subsection{Methods}
		As mentioned, vaccine access seems to not have been a substantial problem by the end of 2021. Instead, vaccination status reflects the willingness of individuals to get vaccinated.
		
		The decision on vaccination can reasonably be broken down into its costs and benefits, and beliefs that agents have about them. We would expect, hence that in areas of lower population density, or where COVID would generally spread less quickly, the benefit of vaccination would be lower. Furthermore, as COVID-19 has a much higher fatality rate among older persons (CITE), they would find a greater benefit to vaccination, and we would expect more vaccination, in counties with more elderly people. As beliefs on vaccination's benefits and supposed harms seem to be affected by, or at least correlated with, religious status, race, and political identity, we also expect these variables.
		
		We begin hence by first running the following cross-sectional regression on the national dataset of 3075 counties:
		
		\begin{equation} \label{eq:crosssection}
			\begin{split}
				\textrm{vaccuptake}_i = &\beta_0 + \beta_1 \textrm{repvotes}_i + \beta_2 \textrm{whiteevangelical}_i + \beta_3 \textrm{catholic}_i \\ 
				&+ \beta_4 \textrm{black}_i + \beta_5 \textrm{poverty}_i + \beta_6 ln(\textrm{medincome}_i) + \beta_7 ln(\textrm{col}_i) \\ 
				&+ \beta_8 \textrm{pop60to79}_i + \beta_9 \textrm{above80}_i + \beta_{10} \textrm{fullcollege}_i + \beta_{11} \textrm{casespc}_i \\
				&+\boldsymbol{\delta}\cdot \mathbf{rural}_i + \boldsymbol{\gamma}\cdot \mathbf{state}_i + u_i
			\end{split}
		\end{equation}
		
		Where $\textrm{vaccuptake}_i$ is a measure of vaccine uptake for a county $i$. Definitions for each covariate are listed in Table \ref{table:definition}. The model was estimated for a cross-section of vaccine uptake by county as of 31 December 2021, and also for vaccine uptake as of 1 June 2021, to see if the effects of the covariates on uptake had changed over time. We measured vaccine uptake as the percentage of county residents who had taken at least two doses of the vaccine, and also an alternatively looked at the percentage of county residents who had taken at least one dose of the vaccine, in case effects were substantially different between the two.
		
		\begin{table}
			\caption{Variable Definitions}
			\begin{minipage}{6.5in}
	\centering
	\def\sym#1{\ifmmode^{#1}\else\(^{#1}\)\fi}
	\def\arraystretch{1}
	\small
	\begin{tabular*}{\textwidth}{@{\extracolsep{\fill}}lp{0.35\linewidth}p{0.4\linewidth}}
	\hline\hline
		&Definition&Justification\\
		\hline
		repvotes&Percentage (0-100) of county vote that Donald Trump received in the 2020 election&Aim to measure effect of political affiliation, which affects news consumption and beliefs on the costs and benefits of vaccination\\
		whiteevangelical&Percentage (0-100) of county identifying as White Evangelical Protestants as of 2020&Aim to measure effect of religion - White Evangelicals are a heavily conservative and Republican-supporting religious group. (CITE)\\
		catholic&Percentage (0-100) of county identifying as Catholics as of 202&Aim to measure the effect of religion - the Pope has actively extolled the benefits of vaccination (CITE), check effect.\\
		black&Percentage (0-100) of county who are African-American as of 2020&Due to historical grievances and mistrust of government, African-Americans are known to be less likely to choose to vaccinate. (CITE)\\
		poverty&Percentage (0-100) of county under the poverty line as of 2019(as defined by the Census)&Those in poverty may have difficulty finding access to vaccines, or more broadly, access to good-quality information on their costs and benefits.\\
		medincome&Nominal median family income for a family of four, as of 2020&Same reason as above. Poverty variable lets us see if effect is stronger at the low end of the income distribution.\\
		col&Cost of living as of 2020, as measured by the EPI.&Allows coefficient on medincome to be interpreted as the effect of changes in real median income (holding prices/cost of living fixed)\\
		pop60to79&Percentage (0-100) of county aged between 60 and 79 in 2019.&Elderly are at greater risk of death and serious illness from COVID-19\\
		above80&Percentage (0-100) of county aged 80 and above in 2019&See above\\
		fullcollege&Percentage (0-100) of county who completed at least 4 years of college, averaged 2015-2019&Those with college degrees may be less susceptible to misinformation on the supposed harms of vaccination\\
		casespc&Total recorded COVID-19 cases as of 1 December 2020, before the start of the vaccination program.&A proxy for how amenable the area is to COVID-19 spread, which would increase the benefit of vaccination against COVID-19\\
		$\mathbf{rural}$&Vector of dummy variables for each possible score on the 2013 Rural-Urban Continuum Code&More rural areas typically have lower population density and hence less risk of catching COVID-19, and so there are lower benefits to vaccination.\\
		$\mathbf{state}$&Vector of state fixed effects&Allows for variation in state policy responses that affected COVID-19 vaccination rates\\
	\hline\hline
	\end{tabular*}
\end{minipage}

			\label{table:definition}
		\end{table}
	
		As the various covariates having different units and standard deviations, interpreting their relative effect sizes (ie, which variables are "more important" in explaining variation in vaccine up take) can be difficult. I hence also estimated the \textit{beta coefficients} of the model - these are calculated by using the z-score of every variable (defined for a covariate $x_j$, with a sample standard deviation $\hat{\sigma}_j$, as $z_j := \frac{x_{ij}-\bar{x}_j}{\hat{\sigma}_j}$)) and noticing then that, if Model (\ref{eq:crosssection}) holds, then it is also true that
		\begin{equation} \label{eq:standardized}
			z_y=\frac{y_i-\bar{y}}{\hat{\sigma}_y} = \sum_{j=1}^{k} \frac{\hat{\sigma}_j}{\hat{\sigma}_y}\beta_j z_{ij}
		\end{equation}
		
		Hence, we can calculate, for each covariate $x_j$, a beta coefficient $\hat{b}_j := \frac{\hat{\sigma}_j}{\hat{\sigma}_y}\beta_j$, interpreted as the standard deviation change in in $y_i=\textrm{vaccuptake}_i$ (in this context) frome a one standard deviation change in covariate $x_j$. This in a sense standardizes the units across the covariates and makes their corresponding effect sizes more directly comparable (although, not perfectly so, as some covariates are drawn from more-spread-out distributions).
		
		We also more explicitly estimate any trends in the covariates over time, by estimating the model
		\begin{equation}
			\textrm{vaccuptake}_{it} =  \beta_0 + \boldsymbol{\alpha}\cdot\mathbf{x}_{i} + \boldsymbol{\beta}\cdot\mathbf{x}_{i}\cdot t + \boldsymbol{\gamma}\cdot\mathbf{x}_{i}\cdot t^2 + u_i
		\end{equation}
		on a panel dataset of observations of vaccine uptake at the start of every month (from May 2021 to March 2021), where $\mathbf{x}_{i}$ is a vector of covariates (including the state fixed effects), using clustered standard errors by county, so as to correct for serial correlation across time in each county. For any covariate $x_j$, we have that
		\begin{equation} \label{eq:trends}
			\frac{\partial \textrm{vaccineuptake}_{it}}{\partial x_{ij}} = \alpha_j + \beta_jt + \gamma_jt^2
		\end{equation}
		Hence, $\beta_j$ and $\gamma_j$ measure a quadratic trend in the effect size of the covariate on the vaccination rate over time.
		
		
		It is important to note that we cannot interpret any of the coefficients estimated in Model (\ref{eq:crosssection}) and Model (\ref{eq:trends})  as estimates of the \textit{causal impact} of the respective covariate on vaccine uptake. That would require an assumption of exogeneity (and strict exogeneity in the panel Model (\ref{eq:trends})), which is unlikely to be satisfied; for example, we cannot conclude, if the coefficient on $ln(\textrm{medincome})$ is positive, that increasing incomes would increase vaccine uptake. Higher incomes may simply correlate with being more informed, an unobserved variable, and so being less likely to believe myths about vaccination harms. This would lead to omitted variable bias, biasing the coefficient on $ln(\textrm{medincome})$ upwards. Instead these coefficients simply reflect correlations in the data, which are at best suggestive.
		
		To answer part (b), however, we need to find out what \textit{causes} vaccine uptake to increase. We hence turn our attention to a policy many states attempted in 2021: vaccine lotteries. We will focus in particular on Colorado's "Colorado Comeback Cash" program, where all those who had received at least one dose of the vaccine were eligible for a weekly \$1 million dollar lottery (CITE). The program began on May 25, and ran to June 30 (CITE).
		
		% Hopefully have more lit review elaboration on this
		
		\begin{figure}
			\centering
			\includegraphics[width=6in]{../graphs/Border_Counties.png}
			\caption*{\footnotesize{Oklahoma, Kansas, Nebraska and Wyoming did not have vaccine lotteries over this period (CITE). New Mexico did, and hence counties bordering New Mexico are excluded from this regression. Utah is excluded from the dataset due to missing data on some covariates. Each county-pair in the pair design dataset consists of one county in Colorado (in dark grey), and one county in an untreated state (in light grey).}}
			\caption{Counties in Pair Design}
			\label{fig:bordercounties}
		\end{figure}
		
		To evaluate the effect of the vaccine lottery, we use a design inspired by Dube (2010) (CITE). We pick counties in Colorado which border counties in states which did not have vaccine lotteries across this period (see Figure \ref{fig:bordercounties}). We reason that these counties are reasonably similar to counties they share a border with (and hence, form a 'county-pair') - most things which affect one county in any period $t$ will probably also affect a paired county across the state border. Table \ref{table:didsummary} compares the selected border counties on some selected covariates; within each county-pair, differences in covariate values are generally within one standard deviation of the covariate.
		
		\begin{table}
			\centering
			\caption{Summary Statistics}
			\centerline{{
\def\sym#1{\ifmmode^{#1}\else\(^{#1}\)\fi}
\begin{tabular}{l*{4}{cc}}
\hline\hline
            &\multicolumn{1}{c}{(1)}&\multicolumn{1}{c}{(2)}&\multicolumn{1}{c}{(3)}&\multicolumn{1}{c}{(4)}\\
            &     Treated&   Untreated&Pair Differences&Nationwide SD\\
            &        mean&        mean&        mean&          sd\\
\hline
repvotes2020pct&       78.58&       84.67&        6.46&       16.03\\
black       &        0.89&        0.53&        0.62&       14.79\\
fullcollege &       21.19&       22.70&        5.71&        9.54\\
cases\_per\_capita&        0.04&        0.06&        0.03&        0.02\\
whiteevangelical&       34.10&       35.31&        5.05&       12.64\\
catholic    &       13.60&       16.69&        5.05&       10.03\\
poverty     &       13.70&       11.38&        3.40&        5.78\\
median\_family\_income\_2020&    63643.50&    66247.97&    10650.24&    16720.16\\
pop60to79   &       21.60&       22.74&        3.66&        4.58\\
above80     &        6.11&        6.84&        2.02&        1.50\\
\hline
\(N\)       &          10&          13&          44&        3075\\
\hline\hline
\end{tabular}
}
}
			\label{table:didsummary}
		\end{table}
		
		However, only one member of each pair (the county within Colorado) is affected by the lottery. We can hence use the paired non-treated county as a control to estimate the causal impact of the vaccine lottery. We construct, as in Dube (2010), a panel of consisting of every cross-border county-pair, from 10 days before the start of the lottery, to 10 days after. We then estimate the model:
		
		\begin{equation} \label{eq:pairdesign}
			\textrm{vaccuptake}_{ipt} = \alpha + \beta \textrm{treated}_{it} + \phi_i + \tau_{pt} + u_{ipt}
		\end{equation}
		
		$i$ is a specific county, $p$ is a county-pair it belongs to, and $t$ is the day in the panel. $\tau_{pt}$ is a pair-specific time fixed effect (shocks beside the lottery happen to both members of a pair) , and $\phi_i$ is a county fixed account (accounting for initial differences). $\textrm{treated}_{it}=1$ if the county $i$ is in Colorado, and $t$ is after 25 May. $\beta$, the effect of the vaccine lottery on vaccine uptake, is consistently estimated if $Cov(\textrm{treated}_{it}, u_{itp})=0$ - shocks which only affect one county in the pair.
		
		Finally, to further check the robustness of this causal estimate, we utilize a Synthetic Control design, as detailed in (CITE) and used in prior studies on Ohio's vaccine lottery (CITE, CITE, CITE). We collapse our 3075 county level dataset into a state-level dataset, and exclude the 23 other states who had vaccine lotteries in this period (CITE). We construct a synthetic Colorado by taking a weighted average of other states to closely match Colorado on some predictive covariates (see CITE for exact details on the estimation method). We use these weights to see how vaccine uptake would have evolved in the synthetic Colorado (made of untreated states), and take the difference between the synthetic Colorado and the real Colorado's vaccine uptake on day $t$ as the causal impact of the vaccine lottery on day $t$, and test its significance.
		
		\section{Results and Discussion}
		
		\begin{table}
			\centering
			\caption{Cross-Section Regression}
			\centerline{{
\def\sym#1{\ifmmode^{#1}\else\(^{#1}\)\fi}
\begin{tabular}{l*{5}{cc}}
\hline\hline
            &\multicolumn{1}{c}{(1)}         &\multicolumn{1}{c}{(2)}&\multicolumn{1}{c}{(3)}         &\multicolumn{1}{c}{(4)}         &\multicolumn{1}{c}{(5)}\\
            &Two Doses, 31 December         &Standardized&One Dose, 31 December         &Two Doses, 1 June         &Standardized\\
            &        b/se         &        beta&        b/se         &        b/se         &        beta\\
\hline
repvotes2020pct&       -0.52\sym{***}&       -0.65&       -0.60\sym{***}&       -0.31\sym{***}&       -0.35\\
            &      (0.03)         &            &      (0.04)         &      (0.03)         &            \\
whiteevangelical&       -0.03         &       -0.03&       -0.05         &       -0.04         &       -0.03\\
            &      (0.06)         &            &      (0.07)         &      (0.05)         &            \\
catholic    &        0.11\sym{*}  &        0.09&        0.14\sym{**} &       -0.02         &       -0.02\\
            &      (0.04)         &            &      (0.05)         &      (0.04)         &            \\
black       &       -0.25\sym{***}&       -0.29&       -0.27\sym{***}&       -0.16\sym{***}&       -0.17\\
            &      (0.03)         &            &      (0.04)         &      (0.03)         &            \\
poverty     &       -0.15         &       -0.07&       -0.14         &       -0.13\sym{*}  &       -0.05\\
            &      (0.08)         &            &      (0.10)         &      (0.06)         &            \\
lmedincome  &        2.22         &        0.04&        1.33         &        4.01         &        0.07\\
            &      (2.96)         &            &      (3.46)         &      (2.28)         &            \\
lcol        &        4.19         &        0.04&        6.37         &        0.41         &        0.00\\
            &      (3.41)         &            &      (4.19)         &      (3.01)         &            \\
pop60to79   &        0.18\sym{*}  &        0.06&        0.20\sym{*}  &        0.31\sym{***}&        0.10\\
            &      (0.08)         &            &      (0.10)         &      (0.06)         &            \\
above80     &        0.73\sym{***}&        0.08&        0.43         &        0.53\sym{**} &        0.06\\
            &      (0.21)         &            &      (0.26)         &      (0.17)         &            \\
fullcollege &        0.10\sym{*}  &        0.07&        0.11\sym{*}  &        0.11\sym{***}&        0.08\\
            &      (0.04)         &            &      (0.05)         &      (0.03)         &            \\
cases\_per\_capita&       57.27\sym{***}&        0.11&       55.28\sym{**} &       52.05\sym{**} &        0.09\\
            &     (15.05)         &            &     (16.87)         &     (16.25)         &            \\
1.rural\_rating&        0.00         &        0.00&        0.00         &        0.00         &        0.00\\
            &         (.)         &            &         (.)         &         (.)         &            \\
2.rural\_rating&        0.20         &        0.00&        1.00         &        1.12\sym{*}  &        0.03\\
            &      (0.55)         &            &      (0.66)         &      (0.54)         &            \\
3.rural\_rating&       -0.20         &       -0.01&       -0.21         &        0.60         &        0.01\\
            &      (0.58)         &            &      (0.71)         &      (0.52)         &            \\
4.rural\_rating&       -0.57         &       -0.01&       -0.92         &        0.76         &        0.01\\
            &      (0.71)         &            &      (0.90)         &      (0.58)         &            \\
5.rural\_rating&        0.08         &        0.00&       -0.47         &        1.54\sym{*}  &        0.02\\
            &      (0.87)         &            &      (1.13)         &      (0.76)         &            \\
6.rural\_rating&       -0.24         &       -0.01&       -0.28         &        0.78         &        0.02\\
            &      (0.63)         &            &      (0.76)         &      (0.56)         &            \\
7.rural\_rating&       -0.38         &       -0.01&        0.10         &        1.07         &        0.03\\
            &      (0.76)         &            &      (0.94)         &      (0.63)         &            \\
8.rural\_rating&       -1.48         &       -0.03&       -2.10         &        0.59         &        0.01\\
            &      (0.92)         &            &      (1.11)         &      (0.79)         &            \\
9.rural\_rating&       -0.96         &       -0.03&       -1.00         &        0.56         &        0.01\\
            &      (0.87)         &            &      (1.06)         &      (0.69)         &            \\
\_cons      &        2.87         &            &        4.21         &      -11.61         &            \\
            &     (41.54)         &            &     (50.46)         &     (33.07)         &            \\
\hline
\(N\)       &        3075         &        3075&        3075         &        3075         &        3075\\
\hline\hline
\end{tabular}
}
}
			\label{table:crosssection}
		\end{table}
		
		Results from the estimation of Model (\ref{eq:crosssection}) are presented in Table \ref{table:crosssection}. Our estimates (not to be interpreted causally, but reflecting correlations within the data) match prior work and have the expected sign - Republicans and black individuals are less likely to get vaccinated, and this is reflected on the county level. Counties with older people, or more college graduates, are more likely to get vaccinated. Poverty and income have the expected signs, and are jointly significant (see caption), although individually insignificant likely due to multicollinearity. Places which had more cases per person in 2020 had higher vaccination rates in 2021, as hypothesized that the benefits of vaccination would be higher and more clear when this was true.
		
		By comparing beta coefficients, we can identify that by far the factor associated with the biggest effect size on vaccine uptake is the republican vote share in the 2020 election, followed by the number of blacks. These effect sizes seem to grow over time, possibly suggesting an increasing hardening of attitudes towards the vaccine.
		
		\begin{table}
			\centering
			\caption{Coefficient Change Over Time}
			\centerline{\begin{minipage}{6.25in}
	\centering
	\def\sym#1{\ifmmode^{#1}\else\(^{#1}\)\fi}
	\def\arraystretch{0.1}
	\newcommand\Tstrut{\rule{0pt}{2.1ex}}         % = `top' strut
	\newcommand\Bstrut{\rule[-0.7ex]{0pt}{0pt}} 
	\begin{tabular*}{\textwidth}{@{\extracolsep{\fill}}l*{2}{c}}
	\hline\hline
	            &\multicolumn{1}{c}{(1)}&\multicolumn{1}{c}{(2)}\Tstrut\Bstrut\\
	            &\multicolumn{1}{c}{Linear Trends}&\multicolumn{1}{c}{Quadratic Trend}\Bstrut\\
	
	\hline
	\multirow{2}{*}{$\textrm{t} \cdot \textrm{repvotes}$}&     -0.0323\sym{***}&     -0.0489\sym{***}\Tstrut\\
	            &   (0.00321)         &   (0.00558)         \\
	[1em]
	\multirow{2}{*}{$\textrm{t} \cdot \textrm{whiteevangelical}$}&     0.00238         &      0.0148         \\
	            &   (0.00515)         &   (0.00792)         \\
	[1em]
	\multirow{2}{*}{$\textrm{t} \cdot \textrm{catholic}$}&      0.0198\sym{***}&      0.0131\sym{*}  \\
	            &   (0.00415)         &   (0.00651)         \\
	[1em]
	\multirow{2}{*}{$\textrm{t} \cdot \textrm{black}$}&     -0.0112\sym{***}&    -0.00437         \\
	            &   (0.00266)         &   (0.00460)         \\
	[1em]
	\multirow{2}{*}{$\textrm{t} \cdot \textrm{poverty}$}&    -0.00860         &     -0.0691\sym{***}\\
	            &   (0.00704)         &    (0.0118)         \\
	[1em]
	\multirow{2}{*}{$\textrm{t} \cdot \textrm{lmedincome}$}&      -0.223         &      -1.213\sym{**} \\
	            &     (0.277)         &     (0.401)         \\
	[1em]
	\multirow{2}{*}{$\textrm{t} \cdot \textrm{lcol}$}&       1.087\sym{***}&       2.204\sym{***}\\
	            &     (0.238)         &     (0.466)         \\
	[1em]
	\multirow{2}{*}{$\textrm{t} \cdot \textrm{pop60to79}$}&     -0.0158\sym{*}  &     -0.0426\sym{***}\\
	            &   (0.00664)         &    (0.0115)         \\
	[1em]
	\multirow{2}{*}{$\textrm{t} \cdot \textrm{above80}$}&      0.0135         &    -0.00855         \\
	            &    (0.0190)         &    (0.0300)         \\
	[1em]
	\multirow{2}{*}{$\textrm{t} \cdot \textrm{fullcollege}$}&    -0.00415         &      0.0173\sym{**} \\
	            &   (0.00414)         &   (0.00643)         \\
	[1em]
	\multirow{2}{*}{$\textrm{t} \cdot \textrm{casespc}$}&       0.417         &       3.850\sym{*}  \\
	            &     (1.271)         &     (1.787)         \\
	[1em]
	\multirow{2}{*}{$\textrm{t}^2 \cdot \textrm{repvotes}$}&                     &     0.00139\sym{**} \\
	            &                     &  (0.000496)         \\
	[1em]
	\multirow{2}{*}{$\textrm{t}^2 \cdot \textrm{whiteevangelical}$}&                     &    -0.00104         \\
	            &                     &  (0.000745)         \\
	[1em]
	\multirow{2}{*}{$\textrm{t}^2 \cdot \textrm{catholic}$}&                     &    0.000559         \\
	            &                     &  (0.000658)         \\
	[1em]
	\multirow{2}{*}{$\textrm{t}^2 \cdot \textrm{black}$}&                     &   -0.000566         \\
	            &                     &  (0.000459)         \\
	[1em]
	\multirow{2}{*}{$\textrm{t}^2 \cdot \textrm{poverty}$}&                     &     0.00504\sym{***}\\
	            &                     &   (0.00108)         \\
	[1em]
	\multirow{2}{*}{$\textrm{t}^2 \cdot \textrm{lmedincome}$}&                     &      0.0825\sym{*}  \\
	            &                     &    (0.0413)         \\
	[1em]
	\multirow{2}{*}{$\textrm{t}^2 \cdot \textrm{lcol}$} &                     &     -0.0931\sym{*}  \\
	            &                     &    (0.0398)         \\
	[1em]
	\multirow{2}{*}{$\textrm{t}^2 \cdot \textrm{pop60to79}$}&                     &     0.00224\sym{*}  \\
	            &                     &  (0.000975)         \\
	[1em]
	\multirow{2}{*}{$\textrm{t}^2 \cdot \textrm{above80}$}&                     &     0.00183         \\
	            &                     &   (0.00274)         \\
	[1em]
	\multirow{2}{*}{$\textrm{t}^2 \cdot \textrm{fullcollege}$}&                     &    -0.00179\sym{**} \\
	            &                     &  (0.000605)         \\
	[1em]
	\multirow{2}{*}{$\textrm{t}^2 \cdot \textrm{casespc}$}&                     &      -0.286         \\
	            &                     &     (0.156)         \\
	\hline
	\(N\)       &       33825         &       33825        \Tstrut\Bstrut \\
	\hline\hline
	\end{tabular*}
    \caption*{\footnotesize{Notes:
		\sym{*} \(p<0.05\), \sym{**} \(p<0.01\), \sym{***} \(p<0.001\)}}
\end{minipage}
}
			\label{table:trends}
		\end{table}
		
		Catholicism was not associated with higher vaccination rates in June, but was by December. The estimation of Model (\ref{eq:trends}), presented in Table \ref{table:trends} affirms this, as in either specification there is an increase in the effect size of Catholicism over time. While we do not have an adequate design to demonstrate this, this may suggest that the Pope's messaging on vaccination (see CITE) has successfully changed attitudes over time. This suggests one way to achieve high vaccination rates across the Untied States: engaging local religious leaders to get them to urge their worshippers to get vaccinated, and to educate them against COVID-19 vaccine related misinformation.
		
		Table \ref{table:trends} also affirms that the Republican and black aversion to vaccination is only getting stronger with time (although, at a declining rate for Republicans, by the coefficient on $t^2\cdot \textrm{repvotes}$). This is again suggestion of polarization, and further highlights the need to break into the Republican and black social networks with pro-vaccine messaging.
		
		\begin{table}
			\centering
			\caption{Effect of Colorado Vaccine Lottery}
			\centerline{\begin{minipage}{6.25in}
	\centering
	\def\sym#1{\ifmmode^{#1}\else\(^{#1}\)\fi}
	\begin{tabular*}{\textwidth}{@{\extracolsep{\fill}} l*{4}{c}}
	\hline\hline
	            &\multicolumn{1}{c}{(1)}&\multicolumn{1}{c}{(2)}&\multicolumn{1}{c}{(3)}&\multicolumn{1}{c}{(4)}\\
	            &\multicolumn{1}{c}{Two Doses}&\multicolumn{1}{c}{One Dose}&\multicolumn{1}{c}{$\Delta \textrm{Two Doses}$}&\multicolumn{1}{c}{$\Delta \textrm{One Dose}$}\\
	\hline
	treated     &      0.0315         &       1.696         &     0.00712         &      0.0332         \\
	            &     (0.116)         &     (1.814)         &    (0.0239)         &    (0.0187)         \\
	\hline
	\(N\)       &        1178         &        1178         &        1178         &        1065         \\
	\hline\hline
	\end{tabular*}
	\caption*{\footnotesize{Notes: Standard errors in parentheses, clustered by state and county-pair. Column 1 shows the estimated treatment effect on the percentage of people who received two doses of the vaccine, within seven days of the start of the treatment. Column 2 shows the estimated treatment effect on the percentage of people who received at least one dose of the vaccine. Columns 3 and 4 measure the effect of the treatment on the daily change in the percentage of people who received both doses and one dose of the vaccine respectively. County fixed effects and pair fixed effects are not reported, for brevity.\\
		\sym{*} \(p<0.05\), \sym{**} \(p<0.01\), \sym{***} \(p<0.001\)}}
\end{minipage}

}
			\label{table:didresults}
		\end{table}
		
		The focus on social messaging and religious factors is then highlighted by our results on the effects of the Colorado vaccine lottery. Regardless of dependent variable chosen, there is no statistically significant effect of the Colorado vaccine lottery on vaccination rates.
		
		
	
		
	\end{spacing}

\end{document}